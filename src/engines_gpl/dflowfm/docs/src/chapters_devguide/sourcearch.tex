\svnid{$Id$}

\chapter{Source tree architecture of \DFLOWFM{}\label{chap:sourcearch}}

\section{Introduction}
This chapter contains a description of the source tree architecture of \DFLOWFM{}, and the development/build environments that can be used on Windows and Linux.

\section{Source builds on Windows: Intel, MS Visual Studio and MSBuild}
Recommended tools:
\begin{itemize}
\item Intel Fortran Composer XE
\item Microsoft Visual Studio (with MS Visual C++)
\end{itemize}

Regular development and manual builds are done within Visual Studio. Batch-mode build on Windows is done via an MSBuild file \file{dflowfm.proj}. Both depend on the solution file \file{dflowfm.sln} discussed in the next section.

\subsection{Solution file, templates and project files}
The list below summarizes the relevant build configuration files, and which settings are where.
\begin{description}
\item \texttt{dflowfm.sln} \\ When making changes to this file, make sure to copy them also into the leading template file \texttt{scripts/template/dflowfm\_template.sln}
%
\item \texttt{scripts/template/dflowfm\_template.sln} \\ The leading solution template file, that is used to produce the actual solution file \texttt{rootdir/dflowfm.sln}
%
\item \texttt{src/*.f90} \\ The majority of all \DFLOWFM{} source files.
\item \texttt{src/dflowfm\_kernel/dflowfm\_kernel.vfproj} \\ The Visual Fortran project file for the kernel as a static library.
\end{description}

The solution file \texttt{dflowfm.sln} is based on a template, such that it can be generated for various combinations of versions of Visual Studio and the Intel Fortran compiler. To generate the actual solution file, in the root dir run:
\begin{Verbatim}[frame=single, framesep=6pt]
python prepare_sln.py
\end{Verbatim}

\section{Source builds on Linux: Intel or GNU, automake and autoconf}
[yet empty]

See \url{http://publicwiki.deltares.nl/display/DFLOWFM/Building+on+Linux}