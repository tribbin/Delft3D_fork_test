\svnid{$Id: bmi.tex 78164 2023-02-14 10:41:59Z dam_ar $}

\chapter{The Basic Model Interface (BMI) for \DFLOWFM}

\section{Introduction}
The \emph{Basic Model Interface (BMI)} is a set of standardized subroutine interfaces that a simulation engine/model may use to implement its API. The BMI implementation in \DFLOWFM{} enables easy access to and interaction with a running \DFLOWFM model schematisation from various host languages, for example Python and C\#.

\section{Basic use of the API}
[yet empty]

\section{Full API description}
[yet empty]

\section{List of available model variables via BMI}
\svnid{$Id: bmi_listofvars.tex 51514 2017-07-06 20:55:24Z mooiman $}  


\begin{longtable}{|>{\ttfamily}p{35mm}|>{\centering $}p{10mm}<{$}|p{\textwidth-70mm-48pt}|p{25mm}|}
\caption{List of available variables via \DFLOWFM{}'s BMI API.}\label{Tab:bmivarlist}\\ [1ex]
\hline
\STRUT \textbf{\rmfamily Variable name} & \centering\textbf{Unit} & \textbf{Description} & \textbf{Shape} \\ [1ex] \hline
\endfirsthead
%\STRUT \textbf{\rmfamily Keyword} & \textbf{Value} & \textbf{Description} & \textbf{Default} \\ [1ex] \hline
%\endfirsthead
%
%
\multicolumn{4}{c}{{\STRUT \tablename\ \thetable{} -- continued from previous page}} \\ [1ex] \hline
\STRUT \textbf{\rmfamily Variable name} & \centering\textbf{Unit} & \textbf{Description} & \textbf{Shape} \\ [1ex] \hline
\endhead
%
\multicolumn{4}{|r|}{{\STRUT continued on next page}} \\ [1ex] \hline
\endfoot
%
\endlastfoot
%
\STRUT

\verb|DFM_COMM_DFMWORLD| & - & The MPI communicator for dflowfm (FORTRAN handle).. & \verb|(/ 0 /)| \\ \hline \STRUT
\verb|iglobal_s| & - & global flow node numbers to help output aggregation later. Should exactly correspond with the original unpartitioned flow node numbers! (as opposed to iglobal). & \verb|(/ ndx /)| \\ \hline \STRUT
\verb|Uorb| & m/s & orbital velocity. & \verb|(/ ndx /)| \\ \hline \STRUT
\verb|twav| & s & wave period. & \verb|(/ ndx /)| \\ \hline \STRUT
\verb|shx| & m & current position. & \verb|(/ nshiptxy /)| \\ \hline \STRUT
\verb|shy| & m & current position. & \verb|(/ nshiptxy /)| \\ \hline \STRUT
\verb|shi| & m & current position. & \verb|(/ nshiptxy /)| \\ \hline \STRUT
\verb|zsp| & m & ship depth at flownodes. & \verb|(/ nshiptxy /)| \\ \hline \STRUT
\verb|shL| & m & ship size L/2, B/2, D  ! for now, fixed max nr =2. & \verb|(/ 2 /)| \\ \hline \STRUT
\verb|shB| & m & ship size L/2, B/2, D  ! for now, fixed max nr =2. & \verb|(/ 2 /)| \\ \hline \STRUT
\verb|shd| & m & ship size L/2, B/2, D  ! for now, fixed max nr =2. & \verb|(/ 2 /)| \\ \hline \STRUT
\verb|stuw| & N & actual thrust force in ship dir. & \verb|(/ 2 /)| \\ \hline \STRUT
\verb|fstuw| & - & thrust setting 0-1. & \verb|(/ 2 /)| \\ \hline \STRUT
\verb|stuwmx| & N & max thrust. & \verb|(/ 2 /)| \\ \hline \STRUT
\verb|roer| & degree & actual rudder angle. & \verb|(/ 2 /)| \\ \hline \STRUT
\verb|froer| & degree & actual rudder setting 0-1. & \verb|(/ 2 /)| \\ \hline \STRUT
\verb|roermx| & degree & max rudder angle. & \verb|(/ 2 /)| \\ \hline \STRUT
\verb|wx| & m/s & wind x velocity   (m/s) at u point. & \verb|(/ lnx /)| \\ \hline \STRUT
\verb|wy| & m/s & wind y velocity   (m/s) at u point. & \verb|(/ lnx /)| \\ \hline \STRUT
\verb|s0| & m & waterlevel    (m ) at start of timestep. & \verb|(/ ndx /)| \\ \hline \STRUT
\verb|s1| & m & waterlevel    (m ) at end   of timestep. & \verb|(/ ndx /)| \\ \hline \STRUT
\verb|a0| & m2 & storage area at start of timestep. & \verb|(/ ndx /)| \\ \hline \STRUT
\verb|a1| & m2 & storage area at end of timestep. & \verb|(/ ndx /)| \\ \hline \STRUT
\verb|vol0| & m3 & volume at start of timestep. & \verb|(/ ndx /)| \\ \hline \STRUT
\verb|vol1| & m3 & volume at end of timestep. & \verb|(/ ndx /)| \\ \hline \STRUT
\verb|hs| & m & waterdepth at cell centre = s1 - bl  (m). & \verb|(/ ndx /)| \\ \hline \STRUT
\verb|ucx| & m/s & cell center velocity, global x-dir (m/s). & \verb|(/ ndkx /)| \\ \hline \STRUT
\verb|ucy| & m/s & cell center velocity, global y-dir (m/s). & \verb|(/ ndkx /)| \\ \hline \STRUT
\verb|ucz| & m/s & cell center velocity, global z-dir (m/s). & \verb|(/ ndkx /)| \\ \hline \STRUT
\verb|sa0| & 1e-3 & salinity (ppt) at start of timestep. & \verb|(/ ndkx /)| \\ \hline \STRUT
\verb|sa1| & 1e-3 & salinity (ppt) at end   of timestep. & \verb|(/ ndkx /)| \\ \hline \STRUT
\verb|satop| & 1e-3 & salinity (ppt) help in initialise , deallocated. & \verb|(/ ndx /)| \\ \hline \STRUT
\verb|tem0| & degC & water temperature at end of timestep. & \verb|(/ ndkx /)| \\ \hline \STRUT
\verb|tem1| & degC & water temperature at end of timestep. & \verb|(/ ndkx /)| \\ \hline \STRUT
\verb|u1| & m/s & flow velocity (m/s)  at   end of timestep. & \verb|(/ lnkx /)| \\ \hline \STRUT
\verb|frcu| & todo & friction coefficient set by initial fields. & \verb|(/ lnx /)| \\ \hline \STRUT
\verb|viusp| & m2/s & user defined spatial eddy viscosity coefficient at u point (m2/s). & \verb|(/ lnx /)| \\ \hline \STRUT
\verb|diusp| & m2/s & user defined spatial eddy diffusivity coefficient at u point (m2/s). & \verb|(/ lnx /)| \\ \hline \STRUT
\verb|kfs| & - & node code flooding. & \verb|(/ ndx /)| \\ \hline \STRUT
\verb|kfst0| & - & node code flooding. & \verb|(/ ndx /)| \\ \hline \STRUT
\verb|ba| & m2 & bottom area, if < 0 use table in node type. & \verb|(/ ndx /)| \\ \hline \STRUT
\verb|bl| & m & bottom level (m) (positive upward). & \verb|(/ ndx /)| \\ \hline \STRUT
\verb|ln| & - & link (2,*) node   administration, 1=nd1,  2=nd2   linker en rechter celnr. & \verb|(/ 2, lnkx /)| \\ \hline \STRUT
\verb|lncn| & - & link (2,*) corner administration, 1=nod1, 2=nod2  linker en rechter netnr. & \verb|(/ 2, lnkx /)| \\ \hline \STRUT
\verb|iadv| & - & type of advection for this link. & \verb|(/ lnx /)| \\ \hline \STRUT
\verb|bob| & m & left and right inside lowerside tube (binnenkant onderkant buis) HEIGHT values (m) (positive upward). & \verb|(/ 2, lnx /)| \\ \hline \STRUT
\verb|vort| & s-1 & vorticity at netnodes. & \verb|(/ ndx /)| \\ \hline \STRUT
\verb|xzw| & m & centre of gravity. & \verb|(/ nump /)| \\ \hline \STRUT
\verb|yzw| & m & centre of gravity. & \verb|(/ nump /)| \\ \hline \STRUT
\verb|xk| & - & Net node x coordinate. & \verb|(/ numk /)| \\ \hline \STRUT
\verb|yk| & - & Net node y coordinate. & \verb|(/ numk /)| \\ \hline \STRUT
\verb|zk| & - & Net node z coordinate. & \verb|(/ numk /)| \\ \hline \STRUT
\verb|kn| & - & Net links: kn(1,:)=from-idx, kn(2,:)=to-idx, kn(3,:)=net link type (0/1/2). & \verb|(/ 3, numl /)| \\ \hline \STRUT
\verb|zbnd1d2d1| & m & 1d2d boundary points 1d water level at new time level. & \verb|(/ nbnd1d2d /)| \\ \hline \STRUT
\verb|zbnd1d2d0| & m & 1d2d boundary points 1d water level at previous time level. & \verb|(/ nbnd1d2d /)| \\ \hline \STRUT
\verb|zcrest1d2d| & m & 1d2d helper array with crest levels. & \verb|(/ nbnd1d2d /)| \\ \hline \STRUT
\verb|edgenumbers1d2d| & m & 1d2d helper array with edge numbers. & \verb|(/ nbnd1d2d /)| \\ \hline \STRUT
\verb|kbnd1d2d| & - & 1d2d boundary points index array. & \verb|(/ 5, nbnd1d2d /)| \\ \hline \STRUT
\verb|width_1d| & m & width 1D SOBEK channel --2D FM coupling. & \verb|(/ nbnd1d2d /)| \\ \hline \STRUT
\verb|qzeta_1d2d| & m3 s-1 & 1d2d output array via BMI for qzeta in 1D SOBEK--2D FM coupling. & \verb|(/ nbnd1d2d /)| \\ \hline \STRUT
\verb|qlat_1d2d| & m3 s-1 & 1d2d output array via BMI for qlat in 1D SOBEK--2D FM coupling. & \verb|(/ nbnd1d2d /)| \\ \hline \STRUT
\verb|qtotal_1d2d| & m3 s-1 & 1d2d output array via BMI for qlat in 1D SOBEK--2D FM coupling. & \verb|(/ nbnd1d2d /)| \\ \hline
\end{longtable}



\section{Adding new model variables to the BMI interfaces}
New model state variables can be made available to the outside world by adding them to the BMI subroutines.
The various BMI-subroutines can be automatically generated using a Python-script and this is strongly advised. Take the following steps:

\begin{enumerate}
\item \textbf{Make the variable available.} Make sure your module variable is \verb|public| (possibly impli\-citly), and pointerable, by adding the \verb|target| attribute (see below).
%
\item \textbf{Add self-describing documentation to the variable.} Behind the variable declaration in your module, add a correct documentation string in the following format:\\
\begin{small}
\begin{alltt}
double precision, allocatable, target :: s1(:) \textit{!< [m] waterlevel at end of }
   \textit{timestep \{"shape": ["ndx"]\}}
\end{alltt}
\end{small}
\ \\
(Don't use newlines in the documentation string.)

The syntax is:
\def\avar#1{\(\langle\)\textsl{#1}\(\rangle\)}
\begin{small}
\begin{alltt}
\avar{type}, target ::\avar{var}(:,\ldots) !< [\avar{unit}] \avar{some description} \{"shape": ["\avar{isize}"%, \ldots ]\}

\end{alltt}
\end{small}
%   \textit{timestep \{"shape": ["ndx"]\}}

%
\item \textbf{Special case for derived type fields} Alternatively, when wanting to expose a particular member field of a user defined type variable as a regular BMI variable, the special prefix \verb|!$BMIEXPORT| is available, in combination with the attribute \verb|"internal"| in the JSON string.

After the actual variable, put one or more comment lines, each one for a single member field that you want to expose. For example:\\
\begin{small}
\begin{alltt}
type(stmtype), target             :: stmpar \linecont \textit{!< All relevant parameters for sediment-transport-morphology module.}
\itshape
!$BMIEXPORT double precision      :: bodsed(:,:) \linecont !< [kg m-2] Available sediment in the bed in flow cell center. \linecont \{"location": "face", "shape": ["stmpar%morlyr%settings%nfrac", "ndx"], \linecont \ "internal": "stmpar%morlyr%state%bodsed"\}
!$BMIEXPORT double precision      :: dpsed(:) \linecont !< [m] Sediment thickness in the bed in flow cell center. \linecont \{"location": "face", "shape": ["ndx"], \linecont \ "internal": "stmpar%morlyr%state%dpsed"\}
\end{alltt}
\end{small}
\ \\
(Don't use newlines in the documentation string, the $\hookleftarrow$s are for readability only.)\\
The specified internal variable expression will be exposed in the BMI under the specified name (for example, \verb|get_var("bodsed")| will return \verb|stmpar%morlyr%state%bodsed|).
%
\item \textbf{Run the BMI-generator script.} Open a DOS-box located at your source scripts dir and run \verb|generate.cmd|:\\
\begin{Verbatim}
$ cd D:\your_dfm_sourcecode\scripts
$ generate.cmd
\end{Verbatim}
\ \\
Note that the \verb|generate.cmd| file contains a short list of FORTRAN files, only those will be scanned for BMI-comments. Add your file if it is not in the list yet.
%
\item \textbf{In case of new modules.} In addition to the automatically generated BMI-interface code parts, some parts require manual editing. Open \verb|unstruc_bmi.F90| and find all subroutines that contain a statement similar to:
\begin{Verbatim}
include "bmi_set/get_var(_shape/_rank/_name/_role/_type).inc"
\end{Verbatim}
\ \\
At the top of subroutines, verify that you module is being made available via a \verb|use| statement.
\end{enumerate}


